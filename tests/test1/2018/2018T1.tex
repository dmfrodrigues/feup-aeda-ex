\setcounter{chapter}{17}
\exam{Teste 1 2018}
\begin{center}
\begin{tabular}{r | l}
	Pergunta & Resposta \\ \hline
	1 & E \\
	2 & C \\
	3 & D \\
	4 & B \\
	5 & A \\
	6 & C \\
	7 & B \\
	8 & C
\end{tabular}
\end{center}
\question{Pergunta 9}
É verdade que o polimorfismo permite que os membros-função sejam específicos ao contexto, uma vez que o principal motivo da existência de polimorfismo é evitar a repetição de código para classes que partilham algumas semelhanças mas são demasiado diferentes para serem sempre considerados a mesma classe, assim como permitir a criação de interfaces para que seja possível interagir com diferentes classes utilizando a mesma chamada de função. Os membros-função podem ser específicos ao contexto porque uma classe base pode implementar uma função como virtual, e uma classe derivada pode implementar a mesma função específica ao seu objetivo.
\subsection{Pergunta 10}
Vantagens da pesquisa binária:
\begin{itemize}
	\item Possui complexidade $O(\log N)$ em vez da complexidade $O(N)$ da pesquisa linear.
\end{itemize}
Desvantagens da pesquisa binária:
\begin{itemize}
	\item É necessária a definição de um operador $<$ (menor) ou $>$ (maior), enquanto pesquisa linear apenas necessita do operador $=$ que geralmente é mais fácil de implementar que $<$, $>$.
	\item É necessário ordenar o vetor, e os algoritmos de ordenação com menor complexidade são $O(N\log N)$.
\end{itemize}
\end{document}
