\setcounter{chapter}{17}
\exam{2018 Teste 2}
\begin{center}
\begin{tabular}{r | l}
	Pergunta & Resposta \\ \hline
	1 & A \\
	2 & D \\
	3 & A \\
	4 & D \\
	5 & C \\
	6 & C \\
	7 & E \\
	8 & B
\end{tabular}
\end{center}
\question{Pergunta 9}
Lista duplamente ligada, uma vez que suporta todas as operações que uma lista simplesmente ligada suporta (e com as mesmas complexidades), além de evitar a necessidade de criar uma nova lista com os elementos em ordem inversa (de complexidade $O(N)$) dado que os elementos da lista também são ligados para trás, pelo que basta percorrer a lista para trás começando no último elemento.
\question{Pergunta 10}
Um vetor permite acesso aleatório aos seus elementos (ou seja, acesso à $i$-ésima posição) em tempo constante, enquanto uma lista permite apenas acesso sequencial (para aceder ao $i$-ésimo elemento é necessário iterar sobre os $i-1$ elementos anteriores).\\
Uma lista admite inserção e remoção de elementos em qualquer parte da estrutura em tempo constante, enquanto um vetor só admite inserção e remoção em tempo constante nos primeiro e último elementos, sendo a inserção e remoção de elementos no meio do vetor operações com complexidade $O(N)$ por implicarem deslocar os elementos subsequentes do vetor.
\end{document}
